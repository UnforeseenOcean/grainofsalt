\documentstyle{article} 
\setlength{\textwidth}{6.5in}
\setlength{\textheight}{9in}
\addtolength{\topmargin}{-0.875in}
\addtolength{\oddsidemargin}{-0.875in}

\begin{document}

\title {Sparse Matrix Package}
\author {Richard Rudell}
\maketitle

\section{General Information}
The package {\em sparse} implements a general sparse-matrix data structure.
This documentation describes the fields of the sparse-matrix structure which
a user is allowed to examine, and the functions which are defined in the
package for manipulating the sparse matrix.


\subsection{Structure Definitions}
This section gives the structure layouts for the fields of the sparse
matrix structures which the user is allowed to reference directly.

\begin{verbatim}
/*
 *  sparse matrix element
 */
typedef struct sm_element_struct sm_element;
struct sm_element_struct {
    int row_num;                        /* row number of this element */
    int col_num;                        /* column number of this element */
    sm_element *next_row;               /* pointers to next and prev row */
    sm_element *prev_row;
    sm_element *next_col;               /* pointers to next and prev column */
    sm_element *prev_col;
};


/*
 *  sparse row
 */
typedef struct sm_row_struct sm_row;
struct sm_row_struct {
    int row_num;                        /* row number of this row */
    int length;                         /* number of elements in this row */
    sm_element *first_col;              /* first element in this row */
    sm_element *last_col;               /* last element in this row */
    sm_row *next_row;                   /* next and previous rows */
    sm_row *prev_row;
};


/*
 *  sparse column
 */
typedef struct sm_col_struct sm_col;
struct sm_col_struct {
    int col_num;                        /* column number of this column */
    int length;                         /* number of elements in this column */
    sm_element *first_row;              /* first element in this column */
    sm_element *last_row;               /* last element in this column */
    sm_col *next_col;                   /* next and previous columns */
    sm_col *prev_col;
};


/*
 *  sparse matrix
 */
typedef struct sm_matrix_struct sm_matrix;
struct sm_matrix_struct {
    sm_row *first_row;                  /* first and last rows in matrix */
    sm_row *last_row;
    sm_col *first_col;                  /* first and last columns in matrix */
    sm_col *last_col;
    int nrows;                          /* number of rows */
    int ncols;                          /* number of columns */
};
\end{verbatim}

\subsection{Iteration Macro's}
\begin{verbatim}

#define sm_foreach_row(A, prow) \
    for(prow = A->first_row; prow != 0; prow = prow->next_row)

#define sm_foreach_col(A, pcol) \
    for(pcol = A->first_col; pcol != 0; pcol = pcol->next_col)

#define sm_foreach_row_element(prow, p) \
    for(p = prow->first_col; p != 0; p = p->next_col)

#define sm_foreach_col_element(pcol, p) \
    for(p = pcol->first_col; p != 0; p = p->next_col)
\end{verbatim}

These simple iteration macros will not operate correctly if the current
element (whether it be a row, column, or element) is deleted during
the loop.  This can be fixed
(manually) using the following sort of construct:

\begin{verbatim}
sm_matrix *A;
sm_row *prow, *prownext;

for(prow = A->first_row; prow != 0; prow = prownext) {
    prownext = prow->next;
        .
        .
        .
}
\end{verbatim}

Note, however, that this modified
iteration construct does not allow the
next item to be deleted during the generation.

\subsection{Matrix Manipulation Functions}

\begin{verbatim}
sm_matrix *
sm_alloc()
\end{verbatim}

        Allocates a new sparse matrix.  Initially it has no rows and no columns.


\begin{verbatim}
void
sm_free(A)
sm_matrix *A;
\end{verbatim}

        Frees a sparse matrix.  All rows, columns and elements of the matrix
        are discarded.


\begin{verbatim}
sm_matrix *
sm_dup(A)
sm_matrix *A;
\end{verbatim}

        Make a copy of the sparse matrix A.


\begin{verbatim}
sm_element *
sm_insert(A, rownum, colnum)
sm_matrix *A;
int rownum, colnum;
\end{verbatim}

        Create an element at (rownum, colnum) in the sparse matrix A, if one
        does not already exit.  Returns a pointer to the sparse matrix element.


\begin{verbatim}
sm_element *
sm_find(A, rownum, colnum)
sm_matrix *A;
int rownum, colnum;
\end{verbatim}

        Locate the item at position (rownum, colnum) in sparse matrix A.
        Returns 0 if no element exists at that location, otherwise returns a
        pointer to the sparse matrix element.


\begin{verbatim}
void
sm_remove(A, rownum, colnum)
sm_matrix *A;
int rownum, colnum;
\end{verbatim}

        Remove the element at (rownum, colnum) in sparse matrix A (if it exists).


\begin{verbatim}
void
sm_remove_element(A, p)
sm_matrix *A;
sm_element *p;
\end{verbatim}

        Remove the element p from the sparse matrix A.  Disaster can occur if
        p is not an element of A.


\begin{verbatim}
sm_row *
sm_get_row(A, rownum)
sm_matrix *A;
int rownum;
\end{verbatim}

        Return a pointer to the specified row of sparse matrix A.  Returns 0
        if there are no elements in that row.


\begin{verbatim}
sm_col *
sm_get_col(A, colnum)
sm_matrix *A;
int colnum;
\end{verbatim}

        Return a pointer to the specified column of sparse matrix A.  Returns 0
        if there are no elements in that column.


\begin{verbatim}
void
sm_delrow(A, rownum)
sm_matrix *A;
int rownum;
\end{verbatim}

        Delete a row from the sparse matrix (if it exists).  All elements in
        the row are discarded.


\begin{verbatim}
void
sm_delcol(A, colnum)
sm_matrix *A;
int colnum;
\end{verbatim}

        Delete a column from the sparse matrix (if it exists).  All elements
        in the column are discarded.


\begin{verbatim}
void
sm_copy_row(A, rownum, row)
sm_matrix *A;
int rownum;
sm_row *row;
\end{verbatim}

        Copy a row to the specified row number of sparse matrix A.


\begin{verbatim}
void
sm_copy_col(A, colnum, col)
sm_matrix *A;
int colnum;
sm_col *col;
\end{verbatim}

        Copy a column to the specified column number of sparse matrix A.


\begin{verbatim}
sm_row *
sm_longest_row(A)
sm_matrix *A;
\end{verbatim}

        Return a pointer to the row in A with the most elements; returns
        0 if there are no rows in A.


\begin{verbatim}
sm_col *
sm_longest_col(A)
sm_matrix *A;
\end{verbatim}

        Return a pointer to the column in A with the most elements; returns
        0 if there are no columns in A.


\begin{verbatim}
int
sm_read(fp, A)
FILE *fp;
sm_matrix **A;
\end{verbatim}

        Reads a sparse matrix from a file.  Format is

\begin{verbatim}
                row_num col_num
                    .
                    .
                    .
                row_num col_num
\end{verbatim}

        Returns 1 on normal termination, 0 on any error.  No attempt is made
        to give the cause for the error.


\begin{verbatim}
void
sm_write(fp, A)
FILE *fp;
sm_matrix *A;
\end{verbatim}

        Writes a sparse matrix to a file which can be read by
        {\bf sm\_read()}.


\begin{verbatim}
void
sm_print(fp, A)
FILE *fp;
sm_matrix *A;
\end{verbatim}

        Debugging routine to print the structural layout of a sparse matrix.


\begin{verbatim}
int
sm_num_elements(A)
sm_matrix *A;
\end{verbatim}

        Returns the number of nonzero elements in the matrix.

\subsection{Row Manipulation Functions}

\begin{verbatim}
sm_row *
sm_row_alloc()
\end{verbatim}

        Allocates a new row vector.


\begin{verbatim}
void
sm_row_free(row)
sm_row *row;
\end{verbatim}

        Frees a row vector and discards its elements.


\begin{verbatim}
sm_row *
sm_row_dup(row)
sm_row *row;
\end{verbatim}

        Creates a new row vector which is a copy of an existing row vector.


\begin{verbatim}
sm_element *
sm_row_insert(row, colnum)
sm_row *row;
int colnum;
\end{verbatim}

        Adds a column to a row vector, if it does not already exist.  Returns
        a pointer to the element.


\begin{verbatim}
sm_element *
sm_row_remove(row, colnum)
sm_row *row;
int colnum;
\end{verbatim}

        Removes an element from a row vector, if it exists.


\begin{verbatim}
sm_element *
sm_row_find(row, colnum)
sm_row *row;
int colnum;
\end{verbatim}

        Finds an element of a row vector, or returns 0 if the element does not
        exist.


\begin{verbatim}
int
sm_row_contains(row1, row2)
sm_row *row1, *row2;
\end{verbatim}

        Returns 1 if row2 structurally contains row1 (that is, row2 has
        an element everywhere row1 does).


\begin{verbatim}
int
sm_row_intersects(row1, row2)
sm_row *row1, *row2;
\end{verbatim}

        Returns 1 if row1 and row2 share a column in common.


\begin{verbatim}
sm_row *
sm_row_and(row1, row2)
sm_row *row1, *row2;
\end{verbatim}

        Returns a row vector containing the elements which row1 and
        row2 have in common.


\begin{verbatim}
int
sm_row_compare(row1, row2)
sm_row *row1, *row2;
\end{verbatim}

        Returns -1, 0, or 1 depending on the lexiographical ordering
        of the two rows.  Suitable for use by st (see st.doc).


\begin{verbatim}
int
sm_row_hash(row1, modulus)
sm_row *row1;
int modulus;
\end{verbatim}

        A hashing function defined on rows which is suitable for use
        by st (see st.doc).


\begin{verbatim}
void
sm_row_print(fp, row1)
FILE *fp;
sm_row *row1;
\end{verbatim}

        Print a row as a sparse vector -- useful for debugging.

\subsection{Column Manipulation Functions}

\begin{verbatim}
sm_column *
sm_col_alloc()
\end{verbatim}

        Allocates a new column vector.

\begin{verbatim}
void
sm_col_free()
\end{verbatim}

        Frees a column vector and discards its elements.


\begin{verbatim}
sm_column *
sm_col_dup(col)
sm_column *col;
\end{verbatim}

        Creates a new column vector which is a copy of an existing
        column vector.


\begin{verbatim}
sm_element *
sm_col_insert(col, rownum)
sm_column *col;
int rownum;
\end{verbatim}

        Adds an element to a column vector.  Returns a pointer to the element.


\begin{verbatim}
void
sm_col_remove(col, rownum)
sm_column *col;
int rownum;
\end{verbatim}

        Removes an element from a column vector, if it exists.


\begin{verbatim}
sm_element *
sm_col_find(column, colnum)
sm_column *column;
int colnum;
\end{verbatim}

        Finds an element of a column vector.  Returns 0 if the element does not
        exist.


\begin{verbatim}
int
sm_col_contains(column1, column2)
sm_column *column1, *column2;
\end{verbatim}

        Returns 1 if column2 contains column1 (that is, column2 has
        an element for every row that column1 does).


\begin{verbatim}
int
sm_col_intersects(column1, column2)
sm_column *column1, *column2;
\end{verbatim}

        Returns 1 if column1 and column2 share a row number in common.


\begin{verbatim}
sm_col *
sm_col_and(col1, col2)
sm_col *col1, *col2;
\end{verbatim}

        Returns a column vector containing the elements which col1 and
        col2 have in common.


\begin{verbatim}
int
sm_col_compare(col1, col2)
sm_column *col1, *col2;
\end{verbatim}

        Returns -1, 0, or 1 depending on the lexiographical ordering
        of the two columns.  Suitable for use by st (see st.doc).


\begin{verbatim}
int
sm_col_hash(col, modulus)
sm_col *col;
int modulus;
\end{verbatim}

        A hashing function defined on columns which is suitable for use
        by st (see st.doc).


\begin{verbatim}
void
sm_col_print(fp, col)
FILE *fp;
sm_column *col;
\end{verbatim}

        Print a column as a sparse vector -- useful for debugging.

\subsection{User-definable Fields}

Associated with each matrix, row, column, and element is a single
generic pointer (called the {\em user-word}) for placing user-defined
information.  This field is initialized to zero, and may be set or
modified by using the routines shown below.  Note that {\bf sm\_get()} and
{\bf sm\_put()} are macros, and hence polymorphic.

When one of the objects in this package are
copied (e.g., {\bf sm\_dup()} or {\bf sm\_row\_dup()}), the
user-word fields are initialized to 0 in the copy.
It is the user's responsibility to also copy the user-word fields, if this
is the desired effect.  Likewise, when a matrix, row, column, or element
is freed, the user-word field is not touched.
It is the user's responsibility
to first walk over the given data type, and free all of the user-words
if this is the desired effect.



\begin{verbatim}
type
sm_get(type, p)
\end{verbatim}

        Returns the {\em user-defined} pointer
        associated with the argument p.  p may
        be either a
        {\bf sm\_matrix *}, {\bf sm\_row *}, {\bf sm\_col *},
        or an {\bf sm\_element *}.


        Typical use:

\begin{verbatim}
                sm_matrix *p;
                struct my_struct *my;
                    .
                    .
                    .
                my = sm_get(struct my_struct *, p);
\end{verbatim}


\begin{verbatim}
void
sm_put(p, val)
\end{verbatim}

        Sets the {\em user-defined} pointer assoicated with
        the argument p.  p may be either a
        {\bf sm\_matrix *}, {\bf sm\_row *}, {\bf sm\_col *},
        or an {\bf sm\_element *}.


\begin{verbatim}
        Typical use:

                sm_matrix *p;
                struct my_struct *my;
                    .
                    .
                    .
                sm_put(p, my);
\end{verbatim}
\end{document}
